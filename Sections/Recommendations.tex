The results of this exploratory investigation provide strong evidence for further targeted research into the incorporation of starch into superhydrophobic films. Logically, replacing the PCC used with eggshell waste as done by \cite{khoo_lim_2017} should be the next step to further improve the sustainability credentials of the novel films developed in this investigation. Attempts to reduce the ethanol used in the formulation protocol, and thereby reduce VOC offgassing during drying,  will also help towards this goal. Additionally, a complete study of the hysteresis (namely $\theta_R$) using sliding, captive bubble or evaporation methods (\cite{eral}) should be undertaken to characterise the films further and further verify the optimised formulation presented here. To verify industrial applicability, characterisation using SEM, FTIR and TGA are required and investigations into alternative coating methods such as spin coating or bio-mineralisation (\cite{tang_chang_li_ge_niu_wang_jiang_sun_2021} will provide further insight into the real-world practicality of these film. More broadly, further starch modifications should be investigated to dampen its hydrophilicity; this includes cross-linking of polymers to occupy hydroxyl groups and etherification/esterification using different materials with the objective of increasing hydrophobicity and decreasing apparent hygroscopicity (\cite{wang}).  

%\begin{itemize}
%\item Biomneralization 
%\item crystallization 
%\item Ball milling to make more uniform powder \cite{fang_2019} rather than pestle mortar
%\item modified starches - use alkyl groups 
%\item Homogenization 
%\item Control Humidity 
%\item use of other surfactants with starch 
%\item Stearic acid - Vivian Consuelo Reolon Schmidt a et al. 
%\item cross- linking from hydrophobic modifications (\cite{wang} 
%\item Synthesis of transparent films \cite{rewritable}
%\item Changing substrate - silanization of glass to make films stick better.
%\item Investigate 2 phase separation of powder and binder due to differing surface energies. Eradicate with spray coating. 
%\item Improve hydrophobic and mechanical properties of starch films by addition of modified fillers. %https://www.sciencedirect.com/science/article/pii/S0141813019354066
%\item Thermal properties (TGA Analysis)
%\item FTIR (Test hydrophobic starches produced) 
%\item SEM (Surface Morphology and particle size of starch)
%\item The failure in measuring the WCA using the Sesile Drop method led us to want to adopt the the Captive bubble method, where the CA of a water-submerged air bubble underneath the test plate was measured following the protocol of Wu [24]. Or evaporation method. 
%\item A modification of the sessile drop method is the evaporation method [29, 32, 33], where the receding angle is measured as a droplet evaporates
%\item cellulose can be used as a replacement. 
%\end{itemize}
%\begin{itemize}
%\item Different concentrations of Ethanol to influence drying rate 
%\item Starch alternatives - we used modified starches that were less hydrophillic - less swelling i.e. OMS Alkyl modified
%\item different processes to reach starch gelatinization (mix longer?)
%\item different natural binder e.g. ethyl cellulose (swelling problem)
%\end{itemize}

