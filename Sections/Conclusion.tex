This paper built on previous work done by \cite{khoo_lim_2017}, \cite{Vatsal} \& \cite{Kokkinis_2017} by replacing the petrochemically derived materials with a biopolymer in corn starch. The modifications successfully produced comparable static and advancing contact angles to the PE/Toluene films in the optimised formulation:

\begin{table} [H]
\centering
\begin{tabular}{llr1}
\toprule
Formulation & Static CA ($^\circ$) & Adv. CA ($^\circ$)\\
\midrule
A-40-2.5   & 154.8 $\pm$ 4.7 & 156.0 $\pm$ 4.3              \\ 
B-50-5.5   & 151.7 $\pm$ 1.5    & 155.3 $\pm$ 3.9           \\ 
\bottomrule
\end{tabular}
\caption{Table comparing Static and Advancing ($\theta_A$) contact angles of previous and optimised formulations}
\label{Conc}
\end{table}
With comparable angles at a first glance, further statistical analysis on contact angles revealed stronger conclusions: B-50-5.5 exhibited no statistical difference in either contact angle when compared to A-40-2.5 and therefore it was a viable option for sustainable development of superhydrophobic films. Durability-wise, the hygroscopicity of starch remained a limitation in DVS experiments. However, there was an insignificant change in hydrophobicity witnessed during the outdoor test; practically, the films had no degradation in integrity or hydrophobicity during this time. Immersion testing further attested to the practical integrity of the films, showing no statistical degradation of films. Mechanical durability is decreased to a limited degree, however. Additionally, starch \emph{modifications} proved unsuccessful; OMS saw difficulty in continuous film formation and the hydrophobicity exhibited by TEOS was outmatched by the optimised formulation containing native corn starch.
Globally, despite the hydrophilic properties of corn starch, there is sufficient evidence to conclude that starch is a promising eco-binder to eliminate PE/Tol in the production of superhydrophobic films using hydrophobic calcium carbonate powder. 


%\begin{itemize}
%    \item Scope to use starch binder to replace A
%    \item Promising range of hydrophobic films and superhydrophobicity achieved through optimisation
%    \item Static angle reported for 'A-40-2.5' 154.8$^\circ$ ± 4.7 $^\circ$ and B-50-5.5 151.7 $^\circ$ ± 1.5 $^\circ$
%    \item Statistical analysis on advancing angles has strong conlusions - ANOVA test has shown that at least one of the starch formulations are different from the %'population' of starch formulations.  Proven by t-test B-50-5.5 statistically different to B-40-5.5 with 95\% certainty hence an optimum verified from Advancing contact %angle as well as static.
%    \item t-test between A-40-2.5 and B-50-5.5  proves we cannot reject null hypothesis and means that starch can be replaced from an advancing angle viewpoint. 
%    \item Hygroscopicity of starch remains a limitation but insignificant effect on hydrophobicity shown from the outdoor test. 
%    \item Outdoor testing has conclusive results, impressive performance considering derived from starch 
%    \item Mechanical durability is decreased to a limited degree, expected from starch comapred to PE. 
%    \item Elimianted toluene and plastic use, semi-renewable and can be combined with renewable sources of CaCO$_3$ such as eggshell so promising. 
%    \item Starch modifications were not successful due to the scientific nature of the report. Cannot change more than one variable, potentially with further scope a %modification can enhance film performance. 
%    \item 
%\end{itemize}

%Despite the properties of starch being heavily influence by water, there is sufficient evidence to conclude that starch is a  promising eco-binder to replace PE/Tol to produce superhydrophobic films using hydrophobic calcium carbonate powder. 