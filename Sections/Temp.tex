Besides the superhydrophobicity of the powder,  there is an additional challenge of producing a sustainable paint (binder solution) so that a homogeneous coating can be practically applied. Due to environmental concerns, starch has received a lot of attention; as a natural biodegradable biopolymer, it is a promising candidate to replace the petroleum-derived materials currently in use. Starch already boasts a wide range of uses in industry - it is used in cosmetics, textiles, pharmaceuticals and fuels amongst others (\cite{kaur_ariffin_bhat_karim_2012}).  Amongst the candidates for petroleum replacement, starch one of the most encouraging due to its diversity, film forming properties, availability, cost-effectiveness, and ease of handling (\cite{dufresne_castaño_2016}, \cite{CAO2008119}).  
\par Starch is obtained from various botanical sources; as a consequence of its numerous sources, starch itself is a diverse polymer with different shapes, sizes, structures and chemical properties (\cite{smith_2001}). Starch contains two macro molecules: amylose (linear) and amylopectin (branched), with the varying compositions allowing the diverse physico-chemical properties of starch-based films. Corn Starch (Zea mays L.) was was sourced with a 28\% w/w amylose content  and a gelatinization temperature of 59.7 °C (\cite{luchese_spada_tessaro_2017}). Gelatinization was essential to mitigate issues with cracking, hygroscopicity and non-continuous/inconsistent film formation (\cite{ZOBEL1984285}). Corn Starch was chosen as a starting point due to its low-cost and relatively high amylose content contributing to improved film flexibility and continuity (\cite{Amylosebenefit}).
\par Starch repeating units feature three free hydroxyl groups on each glycoside ring, making the molecule strongly hydrophilic; as a result, various starch functionalisation processes for hydrophobicity were reviewed to counteract the inherent hydrophilicity of the molecules (\cite{wang}). \cite{jiang_dai_qin_xiong_sun_2016} demonstrated short-chain amylose starch modification in ethanol solution using 2-Octen-1-ylsuccinic anhydride (OSA) to produce amphiphilic OSA-starch-nanoparticles (OSA-SNP’s). The hydrophobicity of OSA-SNP’s increased with the degree of substitution as the hydroxyl groups on starch molecules were replaced by hydrophobic OSA groups. \cite{yu_jiang_zheng_cao_hou_xu_wang_jiang_pan_2019} formulated OSA-SNP powders which were pressed to obtain tablets; the starch modification had little effect on the particle size or morphology of starch, but increased the contact angle significantly, from 25.4° to 70.1°. The amphipathy of modified starch polymers would increase the hydrophobicity of films over those that retained their hydrophilic hydroxyl groups e.g.  unmodified corn starch. The OSA-Corn starch films were created using a modified method as described by \cite{OMSProcess}. 
\par The addition of inorganic tetraethylorthosilicate (TEOS) can also improve hydrophobicity of starch based-films (\cite{TEOS}) by the synergism of the components to form a starch-inorganic hybrid material. An adapted procedure was undertaken to prepare films by dip-coat method with hydrolysis and condensation of TEOS in situ under a controlled pH, as demonstrated by (\cite{TEOS}). 
\par Water molecules can also act as a plasticiser for hydrophilic polymers. \cite{wang_gu_hong_cheng_li_2011} demonstrated how the addition of silica nanoparticles at 10\% w/w corn starch not only increased shear strengths of starch-based binder but also increased the water resistance by 20.2\%, something invaluable for industrial applications in humid environments. Starch films incorporating silica nanoparticles were also investigated.

